%%%%%%%%%%%%%%%%%%%%%%%%%%%%%%%%%%%%%%%%%
% Medium Length Professional CV
% LaTeX Template
% Version 2.0 (8/5/13)
%
% This template has been downloaded from:
% http://www.LaTeXTemplates.com
%
% Original author:
% Trey Hunner (http://www.treyhunner.com/)
%
% Important note:
% This template requires the resume.cls file to be in the same directory as the
% .tex file. The resume.cls file provides the resume style used for structuring the
% document.
%
%%%%%%%%%%%%%%%%%%%%%%%%%%%%%%%%%%%%%%%%%

%----------------------------------------------------------------------------------------
%	PACKAGES AND OTHER DOCUMENT CONFIGURATIONS
%----------------------------------------------------------------------------------------

\documentclass{resume} % Use the custom resume.cls style
\usepackage{color}
\renewcommand{\familydefault}{\rmdefault}
%color packages for timeline graph
\definecolor{dark}{RGB} {24, 69, 59}



\usepackage[left=0.5in,top=0.5in,right=0.5in,bottom=0.5in]{geometry} % Document margins
\usepackage[pdftex]{hyperref}
\newcommand{\tab}[1]{\hspace{.2667\textwidth}\rlap{#1}}
\newcommand{\itab}[1]{\hspace{0em}\rlap{#1}}

\hypersetup{
	colorlinks=true,
	linkcolor=blue,
	filecolor=dark,
	citecolor=dark,    
	urlcolor=black,
}

\begin{document}
\rmfamily
\vspace{.75em}
\begin{tabular*}{\textwidth}{l@{\extracolsep{\fill}}r}
\textbf{\href{https://raghavrakesh.com/}{\LARGE {\color{dark}Raghav Rakesh}}} & {} \vspace{.5em} \\
{Michigan State University} &{}\\
{Department of Economics} &  Website:
\href{https://raghavrakesh.com/}{{\color{dark}raghavrakesh.com}}  \\
{486 W. Circle Drive} & Email: \href{mailto:rakeshra@msu.edu}{{\color{dark}rakeshra@msu.edu}}  \\
{East Lansing, Michigan - 48824} & Phone: +1-517-575-5206

\end{tabular*}

\vspace{0.5em}
%----------------------------------------------------------------------------------------
%	EDUCATION SECTION
%----------------------------------------------------------------------------------------
\vspace{-1em}
\begin{rSection}{Education}

{\bf Michigan State University} \hfill {May 2024 (Expected)} \\ 
Ph.D. in Economics \hfill {}

{\bf Michigan State University} \hfill {May 2020} \\
 M.Sc. in Economics \hfill {}

{\bf Delhi School of Economics, University of Delhi} \hfill {May 2015} \\ 
M.A. in Economics\hfill {}

{\bf Hans Raj College, University of Delhi} \hfill {May 2013}\\ 
B.A.(Honours) in Economics \hfill {}
\end{rSection}

%----------------------------------------------------------------------
%	TECHNICAL STRENGTHS SECTION
%-----------------------------------------------------------------------


\begin{rSection}{References}
	\vspace{.5em}
	\begin{minipage}{0.3\linewidth}
		{\bf Todd Elder (Chair)}\\
		Department of Economics\\
		Michigan State University\\
		+1 (517)-355-0353\\
		\href{mailto:telder@msu.edu}{telder@msu.edu}\\
	\end{minipage}
	\hfill
	\begin{minipage}{0.3\linewidth}
		{\bf Ben Zou} \\
		Department of Economics\\
		Purdue University\\
		+1 (765)-496-2621\\
		\href{mailto:zou136@purdue.edu}{zou136@purdue.edu}\\
	\end{minipage}
	\hfill
	\begin{minipage}{0.38\linewidth}
		{\bf Nishith Prakash}\\
		School of Public Policy and Urban Affairs \\
		Northeastern University\\
		+1 (617)-373-6228\\
		\href{mailto:n.prakash@northeastern.edu}{n.prakash@northeastern.edu}\\
	\end{minipage}


	
\end{rSection}



%----------------------------------------------------------------------------------------
%	WORK EXPERIENCE SECTION
%----------------------------------------------------------------------------------------

\begin{rSection}{Research Fields}\itemsep -2pt
	
	{\bf Primary:} Labor Economics, Economics of Education\\
	{\bf Secondary:} Development Economics, Urban Economics
	%-----------------------------------------------
	
\end{rSection}



%----------------------------------------------------------------------------------------
\begin{rSection}{Job Market Paper}
	\vspace{.2em}
	%\href{https://raghavrakesh.com/files/Disruptive\%20Interactions_JMP_Anjali\%20P\%20Verma.pdf}{\bf {\color{dark}Disruptive Interactions: Long-run Peer Effects of Disciplinary Schools}} {\bf (Job Market Paper)}
	
	{\bf {\color{dark}International Students in Higher Education And Domestic Students' Outcomes}} {\bf (Job Market Paper)}
	
	
	Recent decades witnessed a rapid increase in foreign post-secondary student enrollment in the US, substantially altering the college student landscape. While evidence suggests that foreign students contribute significantly to university revenue and the host economy, there is a lot of debate around their impact on domestic students' outcomes. Using rich administrative and survey data of students from a large US public university, this paper explores the effects of exposure to foreign peers in college courses on domestic students' academic outcomes. To estimate the causal effects, I focus on first-term introductory math courses and leverage idiosyncratic variation in the share of foreign peers across terms but within the course-instructor. I find that exposure to foreign peers in lower-ability (non-calculus) courses has a sizable negative effect on the graduation rate of domestic students; students in higher-ability (calculus-based) courses are unaffected by their foreign peers. The decline in graduation comes through a drop in students graduating with non-STEM degrees, with no impact on the number of STEM graduates. Further, the negative effects are incurred by domestic students of all races except Asians; domestic Asian students incur positive effects. Exploring potential mechanisms, I find suggestive evidence that relative performance, limited interaction, lack of shared interests or culture, and language barrier between domestic and foreign students may be driving the results. At the same time, I do not find evidence that ability differences or social preferences associated with races or nationalities lead to the effect. 
	
	\vspace{.5em}
\end{rSection}



\begin{rSection}{Publications}
	\vspace{.5em}
	\href{https://onlinelibrary.wiley.com/doi/abs/10.1111/ecin.13044}{\bf {\color{dark}Science Education and Labor Market Outcomes in a Developing Economy}} \\
	Joint with Tarun Jain, Abhiroop Mukhopadhyay and Nishith Prakash, \textbf{Economic Inquiry}, 60(2), 741-763, April 2022
	
	
	We examine the association between studying science in higher secondary school and labor market earnings in India. Studying science in high school is associated with 22\% greater earnings than studying business or humanities. Earnings for science students are further enhanced with some fluency in English. Science education is also associated with more years of education, completing a professional degree, returns to entrepreneurship and working in public sector positions. Primary survey of high school students shows no discernible differences in behavioral characteristics of science students compared to others.
	
	\vspace{.5em}
\end{rSection}

%----------------------------------------------------------------------------------------
\begin{rSection}{Working Papers}
	\vspace{.5em}
	
	\href{}{\bf {\color{dark}The Local Economic Impacts of Foreign Students}}
	
	Do foreign students affect the economic outcomes of the natives in places with post-secondary institutions? I address this question by examining the impacts of demand shocks induced by expansions in foreign post-secondary student enrollment in the US between 2004 and 2016. Using an instrumental variables strategy that exploits spatial variation in foreign student enrollment expansion over this period, I estimate the causal effects on a vector of local economic outcomes. On average, the demand shocks substantially increased local employment and wages while having no significant effect on housing rent. At the same time, I find no evidence of adverse spillover effects on neighboring areas without post-secondary institutions. Further, the effect on employment increases with population density. However, the effect on housing rent also increases, likely due to limited supply in densely populated areas. The results suggest welfare gains for natives, especially in less densely populated areas that depend heavily on the education sector. While the effect of changes in foreign student enrollment on the local economy is sizable, the effect of changes in domestic student enrollment is small during the same period.
	
	\vspace{.5em}
	
	
\end{rSection}

%----------------------------------------------------------------------------------------


\begin{rSection}{SELECTED WORKS IN PROGRESS}
\vspace{.5em}


{\bf Sowing the Seeds of Entrepreneurship: Evaluation of Entrepreneurial Mindset Development Program in India}\\
Joint with Sofia Amaral, Aakash Bhalothia, Ritam Chaurey, Isis Gaddis, Gaurav Khanna, Samreen Malik, Abhiroop Mukhopadhyay, Nishith Prakash 

\vspace{.5em}

{\bf Weather and College Student Achievement}\\
Joint with Andrew Earle


\vspace{.5em}


{\bf Shaping Minds: The Transformative Effects of Theatre-Based Learning}\\
Joint with Ritam Chaurey, Sara Constantino, Shantanu Khanna, Abhiroop Mukhopadhyay, Nishith Prakash, Raisa Sherif


\end{rSection}

%--------------------------------------------------------------

\begin{rSection}{Research Grants}
	\begin{rSubsection}{}{}{}{}
		\item[] \textbf{J-PAL Learning for All Initiative}: \$83,044 --- ``Shaping Minds: The Transformative Effects of Theatre-Based Learning'', 2024-2025 (Co-PI) \\
		
		\item[] \textbf{YALE-RISE}: \$300,000 --- ``Entrepreneurial Mindset Development Program in Andhra Pradesh'', 2023-2026  (Co-PI) \\
		
		\item[] \textbf{World Bank - South Asia Gender Innovation Lab}: \$100,000 --- ``Impact Evaluation of the Entrepreneurial Mindset Development Program in India'', 2023-2024 (Co-PI) 
		
	\end{rSubsection}
\end{rSection}


%--------------------------------------------------------------

%-----------------------------------------------------------------
%----------------------------------------------------------------------------------------

\begin{rSection}{Teaching and Research Experience} \itemsep -2pt
\vspace{.5em}
\begin{rSubsection}{Instructor, Michigan State University}{}{}{}
\item[] \quad Intermediate Microeconomics \hfill{2020}

% Summer 2020

\end{rSubsection}


\begin{rSubsection}{Teaching Assistant, Michigan State University}{}{}{}
\item[] \quad Advanced Microeconomics: Game Theory\hfill{2023}
% Spring 2023	
\item[] \quad Introductory Microeconomics ($\times$5) \hfill{2019-2023}
% Fall 2019, Spring 2020, Fall 2021, Spring 2022, Fall 2023
\item[] \quad Intermediate Microeconomics ($\times$3) \hfill{2018-2021}
% Fall 2018, Fall 2020, Spring 2021
\item[] \quad Intermediate Macroeconomics\hfill{2019}
% Spring 2019
\end{rSubsection}


\begin{rSubsection}{Tutor, AEA Summer Program}{}{}{}
	\item[] \quad Advanced Mathematical Methods\hfill{2020}
	% Summer 2020	
	\item[] \quad Advanced Econometrics \hfill{2020}

\end{rSubsection}


\begin{rSubsection}{Research Assistant}{}{}{}
\item[] \quad Research Assistant, Prof. Abhiroop Mukhopadhyay, ISI Delhi \hfill{April 2016-May 2018}
\item[] \quad Reserach Assistant, Prof. Nishith Prakash, University of Connecticut, Storrs \hfill{April 2016-May 2018}
\end{rSubsection}


\end{rSection}


%--------------------------------------------------------------
\vspace{0.5em}

\begin{rSection}{Professional Activities}
\vspace{.5em}

\begin{rSubsection}{Referee}{}{}{}
\item[] \quad  Economics of Education Review, Economic Modelling, PLOS One
\end{rSubsection}


\begin{rSubsection}{Conference Presentations}{}{}{}

\item[] \quad Southern Economic Association Annual Meeting \hfill{2023(scheduled), 2022}
\item[] \quad Midwest Economic Association Annual Meeting \hfill{2023}
\item[] \quad 15th Annual Conference on Economic Growth and Development, ISI Delhi \hfill{2022}
\item[] \quad Research Seminars in Advanced Topics in Economics, Michigan State University \hfill{2022}
\item[] \quad Red Cedar Conference, Michigan State University \hfill{2021}
\end{rSubsection}

\begin{rSubsection}{Invited Seminars}{}{}{}
	
	\item[] \quad Indira Gandhi Institute of Development Research  \hfill{2022}
	
\end{rSubsection}

\begin{rSubsection}{Editor}{}{}{}
	
	
	\item[] \quad Students' Journal (Eostre), Department of Economics, Delhi School of Economics  \hfill{2013-15}

\end{rSubsection}
\end{rSection}

\vspace{0.5em}
%--------------------------------------------------------------

\begin{rSection}{Awards and Fellowships}
\begin{rSubsection}{}{}{}{}
\item[] Whitledge Endowment Fellowship, Michigan State University \hfill{2023}	
\item[] Research Fellowship, Michigan State University \hfill{2022, 2021}
\item[] Supplemental Support Fellowship, Michigan State University \hfill{2019}
\item[] Rank 19 (out of 30,000 students), Regional Mathematics Olympiad, National Board of \\ Higher Mathematics, Government of India \hfill{2009-10}
\item[] National Merit List, Grade 10 Exam, Central Board of Secondary Education, Government of India \hfill{2008}
\end{rSubsection}
\end{rSection}


%--------------------------------------------------------------
\vspace{2em}


\begin{rSection}{Industry Work Experience} \itemsep -2pt
	\vspace{.5em}
		\item[]	{\bf Consultant, Deloitte India LLP} \quad  \hfill{June 2015-March 2016} \\
		Education \& Skill Development, Public Sector Practice
	\begin{itemize}
		\item[-] Undertook due diligence for clients (National and bilateral agencies) to assess the business, financial, and technical viability of 7+ private sector organizations for debt investment
		\item[-] Conducted research and prepared 2+ reports on skill supply-demand gaps for National agencies and Industry bodies \\
	\end{itemize}
	\item[]	{\bf Intern, Indian Credit Rating Agency} {(A Moody's Investors Service Company)} \quad  \hfill{June-August 2014}
	
	\item[]	{\bf Intern, Ernst \& Young} \quad  \hfill{June-August 2012}
			
	
\end{rSection}

\vspace{0.5em}
\begin{rSection}{Technical skills}
	\vspace{.5em}
{\bf Languages/Software:} Stata, Python, SQL, LaTeX, QGIS, GitHub \\ 
{\bf Tools:} Panel Data Econometrics, Causal Inference, Applied Statistics, Randomized Control Trial, Lab Experiment, Survey Design, Machine Learning
\end{rSection}



%----------------------------------------------------------------------------------------


\vspace{0.5em}
\begin{rSection}{Personal Information}
	\vspace{.5em}
	{\bf Citizenship:} India (On F1 visa) \\ 
	{\bf Gender:} Male \\
	{\bf Languages:} English, Hindi (Native) \\~\\
\end{rSection}



%----------------------------------------------------------------------------------------
\vspace{1em}


\textit{Last updated on: September 27, 2023}

\end{document}
